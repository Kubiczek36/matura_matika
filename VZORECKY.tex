\documentclass[10pt,a4paper]{article}
\usepackage[utf8]{inputenc}
\usepackage[czech]{babel}
\setlength{\parindent}{0cm}
\usepackage[T1]{fontenc}
\usepackage{amsmath}
\usepackage{amsfonts}
\usepackage{amssymb}
\usepackage{graphicx}
\usepackage{lmodern}
\pagestyle{empty}
\usepackage{wrapfig}
%\usepackage{fourier}
\usepackage[left=0.8cm,right=0.8cm,top=0.5cm,bottom=0.5cm]{geometry}
\author{Jakub Dokulil}
\newcommand{\nazv}[1]{{\large \textbf{\textsf{#1}}}}
\newcommand{\tg}{\mathrm{tg}}
\newcommand{\cotg}{\mathrm{cotg}}
\title{Vzorečky do matiky}
\begin{document}
\begin{center}
\begin{Large}
\textsc{VZORCE DO MATEMATIKY}
\end{Large}
\end{center}
\hspace{4pt}\nazv{Konstanty:}

\begin{tabular}{lllllllll}
$11^2=121$&$14^2=196$&$17^2=289$&$20^2=400$&$2^2=4$&$2^5=32$&$2^8=256$
\\
$12^2=144$&$15^2=225$&$18^2=324$&$25^2=625$&$2^3=8$&$2^6=64$&$2^9=512$
\\
$13^2=169$&$16^2=256$&$19^2=361$&	&$2^4=16$&$2^7=128$&$2^{10}=1024$
%
\end{tabular}


\begin{tabular}{p{9.5cm}p{9.5cm}}
\multicolumn{2}{l}{
\nazv{Základní algebraické vzorce:}}\\
$(a+b)^2=a^2+2ab+b^2$ & $a^2-b^2=(a+b)(a-b)$ \\ 

$(a+b)^3=a^3+3a^2b+3ab^2+b^3$ & $a^3-b^3=(a-b)(a^2+ab+b^2)$ \\ 
$(a+b+c)^2=a^2+b^2+c^2+2ab+2ac+2bc$ &$a^3 + b^3 =(a+b)(a^2-ab+b^2)$   \\další jdou odvodit pomocí bin. věty&$a^5+b^5=(a+b)(a^4-a^3b+a^2b^2-ab^3+b^4)$\\
$a^n+b^n=(a+b)(a^{n-1}-a^{n-2}b+a^{n-3}b^2-\cdots+b^{n-1})$ \newline  
pro lichá $n$.&$a^n-b^n=(a-b)(a^{n-1}+a^{n-2}b+a^{n-3}b^2 +\cdots +b^{n-1})$ \newline pro $n\in\mathbb{N}$ \\
\end{tabular}
 
\begin{tabular}{p{5cm}p{5cm}}

\multicolumn{2}{l}{\nazv{Planimetrie:}}\\
\multicolumn{2}{l}{\textbf{Čtverec}}\\
Obvod, obsah&$o=4a$, $S=a^2$\\
%Obsah&$S=a^2$\\
Délka úhlopříčky& $u=a\sqrt{2}$\\
\multicolumn{2}{l}{\textbf{Obdélník}}\\
Obvod, obsah&$o=2(a+b)$, $S=ab$\\
%Obsah&$S=ab$\\
\multicolumn{2}{l}{\textbf{Rovnoběžník:}}\\
Obvod&$o=2(a+b)$\\
Obsah&$S=a\cdot v_a=ab\sin\alpha$\\
\multicolumn{2}{l}{\textbf{Trojúhelník:}}\\
%\multicolumn{2}{l}{\includegraphics[width=3cm]{trojuhelnik.pdf}}\\
%$\alpha$ -- vnitřní úhel& $\alpha '$ -- vnější úhel\\
%%příčka v $\vartriangle$& \\
\multicolumn{2}{p{10cm}}{\textbf{Výška} je kolmice spuštěná z vrcholu na protější stranu. \textbf{Těžnice} je spojnice vrcholu a středu protilehlé strany. \textbf{Vnější úhel} je vedlejším úhlem k vnitřnímu. \textbf{Ortocentrum} je průsečík výšek. \textbf{Těžiště} je průsečík těžnic. \textbf{Střední příčka} je spojnicí středů dvou stran, má poloviční délku jak strana třetí. Střed \textbf{kružnice vepsané} se nachází v průsečíku os úhlů. Střed \textbf{kružnice opsané} v průsečíku os stran.}\\
\multicolumn{2}{p{9.5cm}}{Obsah$$S=\frac{a\cdot v_a}{2}=\frac{b\cdot v_b}{2}=\frac{c\cdot v_c}{2}$$}\\
Délka výšky v rovnostranném $\vartriangle$:&$v_a=a\cdot\frac{\sqrt{3}}{2}$\\
\multicolumn{2}{l}{\textbf{Věty o pravoúhlém trojúhelníku}}\\
\textbf{Pythagorova věta:}& $a^2+b^2=c^2$\\
\textbf{Euklidova věta o výšce:}& $v^2=c_a\cdot c_b$\\
\textbf{Euklidova věta o odvěsně}& $a^2=c\cdot c_a$\\\textbf{Thaletova věta:}& Obvodový úhel sestrojený nad průměrem je vždy pravý. \\
\multicolumn{2}{l}{\textbf{Lichoběžník}}\\
Obvod&$o=a+b+c+d$\\
Obsah& $S=\frac{1}{2}(a+c)\cdot v$\\
\multicolumn{2}{l}{\textbf{Pravidelný $n$-úhelník}}\\
Počet úhlopříček& $\frac{n(n-3)}{2}$\\
Vnitřní úhel& $\alpha=\frac{(n-2)\cdot 180^{\circ}}{n}$\\
\textbf{Kruh}&\\
Obvod, obsah& $o=\pi d=2\pi r$, $S=\pi r^2$\\
%Obsah& $S=\pi r^2$\\
\end{tabular}%%%%%%%%%%%%%%%%%%kv. rce + zbytek
\begin{tabular}{ll}
\multicolumn{2}{p{8cm}}{\nazv{Kvadratická rovnice} ve tvaru $ax^2+bx+c=0$ má kořeny:
$$x_{1,2}=\frac{-b\pm\sqrt{b^2-4ac}}{2a}$$}\\
\multicolumn{2}{l}{\nazv{Stereometrie}}\\
\multicolumn{2}{l}{$S_p$ je obsah podstavy, $S_{pl}$ je obsah pláště.}\\
\multicolumn{2}{l}{\textbf{Krychle:}}\\
Povrch&$S=6a^2$\\
Objem& $V=a^3$\\
Tělesová úhlopříčka&$u=a\sqrt{3}$\\
\multicolumn{2}{l}{\textbf{Kvádr:}}\\
Povrch&$S=2(ab+ac+bc) $\\
Objem&$V=abc$\\
Tělesová úhlopříčka&$u=\sqrt{a^2+b^2+c^2}$\\
\multicolumn{2}{l}{\textbf{Hranol:}}\\
Povrch&$S=2S_p + S_{pl}$\\
Objem&$V=S_p\cdot h$\\
\multicolumn{2}{l}{\textbf{Jehlan:}}\\
Povrch&$S=S_p + S_{pl}$\\
Objem&$V=\frac{1}{3}S_p v$\\
\multicolumn{2}{l}{\textbf{Válec:}}\\
Povrch&$S=2\pi r^2 + 2\pi r h$\\
Objem&$V=\pi r^2 h$\\
\multicolumn{2}{l}{\textbf{Kužel:}}\\
Povrch&$S=S_p + S_{pl}=\pi r^2 + \pi r s$\\
Objem&$V=\frac{1}{3}\pi r^2 v$\\
\textbf{Koule}&\\
Povrch& $S=4\pi r^2$\\
Objem& $V=\frac{4}{3}\pi r^3$%%%
%%%%%začátek vět o troj
%%%
\end{tabular}

\vspace{-0.5cm}
\begin{tabular}{p{.52\textwidth}}
\hspace{-0.2cm}\begin{tabular}{p{5cm}p{4.4cm}}\\

&\begin{footnotesize}
%Pokračování v druhém sloupci$\nearrow$
\end{footnotesize}
\end{tabular}

\nazv{Goniometrické funkce:}
%\begin{figure}

%\end{figure}
Hodnoty funkce sinus lze odečítat na $y$-ové ose, u kosinu na $x$-ové ose, 
hodnoty funkcí tangens a kotangens na \uv{posunutých} osách.


Definice funkcí tangens a kotangens:
$$	\tg x=\frac{\sin x}{\cos x}\qquad\cotg x=\frac{\cos x}{\sin x} $$
\textbf{Vzorečky:}

\begin{tabular}{p{.2\textwidth}p{.2\textwidth}}$ \tg x \cdot\cotg x=1$&$ \sin^2 x+\cos^2 x=1$\\$ \sin 2x=2\sin x\cos x$&$\cos 2x=\cos^2 x - \sin^2 x $\\
\end{tabular}

V pravoúhlém trojúhelníku platí:
$$ \sin\alpha=\frac{\text{protilehlá\,odvěsna}}{\text{přepona}}\qquad \cos\alpha=\frac{\text{přilehlá\,odvěsna}}{\text{přepona}}$$
$$\tg\alpha=\frac{\text{protilehlá\,odvěsna}}{\text{přilehlá\,odvěsna}} \qquad\cotg\alpha=\frac{\text{přilehlá\,odvěsna}}{\text{protilehlá\,odvěsna}}$$
%Je vhodné vědět:
%$$\lim\limits_{x\to 0}\frac{\sin x}{x}=1 \Rightarrow \lim_{x\to 0}\frac{\tg x}{x}=1 $$
%\begin{displaymath}
%\end{displaymath}
%$ \limits {\to 0}$


\end{tabular}
\begin{tabular}{l}
\includegraphics[width=.41\textwidth]{gon_fce.pdf}
\end{tabular}

\newpage

\begin{tabular}{p{0.4\textwidth}p{0.3\textwidth}|p{0.3\textwidth}}
\nazv{Derivace funkcí:}
Definice derivace funkce:
$$f'(x)=\frac{\Delta f(x)}{\Delta x}=\lim_{\Delta x\to 0}\frac{f(x+\Delta x)-f(x)}{\Delta x} $$
Nechť jsou funkce $u$, $v$, konstanta $k$.
&
\begin{tabular}{p{2cm}p{2cm}}
$f(x)$& $f'(x)$\\
$u\pm v$&$u'\pm v'$\\
$u\cdot v$& $u'v+uv'$\\
$\frac{u}{v}$&$ \frac{u'v-uv'}{v^2}$\\
$x^n $& $n\cdot x^{n-1}$\\
$ $& $ $\\
\end{tabular}&\begin{tabular}{p{2cm}p{2cm}}
$f(x)$& $f'(x)$\\
$k$&$0$\\
$u\cdot v$& $u'v+uv'$\\
$\frac{u}{v}$&$ \frac{u'v-uv'}{v^2}$\\
$x^n $& $n\cdot x^{n-1}$\\
$ $& $ $\\
\end{tabular}

\end{tabular}

\nazv{Posloupnosti a řady:}

\begin{tabular}{p{5.5cm}p{5.5cm}p{5.5cm}}
&\textbf{Aritmetická posloupnost}&\textbf{Geometrická posloupnost}\\
Rekurentní vzorec& $a_n=a_{n-1}+d $&$a_n=a_{n-1}\cdot d$\\
Vzorec pro výpočet $n$-tého členu&$a_n=a_1+(n-1)d $&$a_n=a_1\cdot q^{n-1} $\\
Vzorec pro součet $n$ členů&$$s_n=\frac{n(a_1+a_n)}{2}$$&$$s_n=a_1\frac{1-q^n}{1-q}$$\end{tabular}


\end{document}
%%%%%%%%%%%%%%%%%%%%%%%%%%%%%%%%%%%%%%%%%%%%%%%%%%%%%%%%%%
%%%%%%%%%%%%%%%%%%%%%%%%%%%%%%%%%%%%%%%%%%%%%%%%%%%%%%%%%%

%%%%%%%%%%%%%%%%%%%%%%%%%%%%%%%%%%%%%%%%%%%%%%%%%%%%%%%%%%

%%%%%%%%%%%%%%%%%%%%%%%%%%%%%%%%%%%%%%%%%%%%%%%%%%%%%%%%%%


%%%%%%%%%%%%%%%%%%%%%%%%%%%%%%%%%%%%%%%%%%%%%%%%%%%%%%%%%%